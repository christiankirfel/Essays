% Turabian Formatting for Research Papers Template, 2018/08/06
%
% Developed using the turabian-formatting package (2018/08/01), available through CTAN: http://www.ctan.org/pkg/turabian-formatting
%
% Additional document class formatting options:
%
% raggedright: ragged right formatting without hyphenations
% authordate: support for the author-date citation style
% endnotes: support for endnotes



\documentclass{turabian-researchpaper}


\usepackage[utf8]{inputenc}
\usepackage{csquotes, ellipsis}

% Specify paper size with geometry package
\usepackage[pass, letterpaper]{geometry}

% For citations, use the biblatex-chicago package
\usepackage{biblatex-chicago}
\addbibresource{works-cited.bib}

% Information for title page
\title{Politische Stellungnahme}
\subtitle{Oder warum ich mich rechts schimpfen lasse.}
\author{Christian Kirfel}
\date{\today}


\begin{document}

\maketitle


\section{Einleitung}

Wir leben in einer Zeit, in der die westliche Kultur vor einem Wandel steht. Wir leben in einer Zeit, in der der Deutsche immer oefter direkt und im Alltag mit den Kontroversen seiner Welt konfrontiert wird. Lange Jahren lebten wir mit abstrakten Problemen. Krieg, im nahen aber doch recht weit entfernten Osten, eine Finanzkrise, die uns nur schwach traf oder den Kampf um die Atomkraft. Hier war es leicht jede Stellung zu beziehen, endlos zu diskutieren und sinnvolle Argumente fuer beide Parteien zu finden. Aber am aller wichtigstens, diese Konflikte waren unserem Alltag fern genug, um sie gut und gerne ignorieren zu koennen. Wurde der Diskurs zu vehement, fuehlte man sich verletzt oder nicht ausreichend repraesentiert, konnte man leicht die Augen verschliessen, den Sender wechseln oder das Thema auf dem Familienfest. Das Leben ging mehr oder weniger unbeindruckt weiter.

Politische rechtschaffenheit gab es auch zur Genuege im Angebot, denn jede rechte Bewegung, bot nichts als Zielscheiben und stets genug Lacher fuer Satire.

Nun aber werden immer mehr Menschen mit den Problemen unserer Zeit direkt konfrontiert. Sei es persoenlich als Verwaltungsangestellter oder Polizist, sei es indirekt ueber die Berichte vertrauenwuerdiger Verwandter oder sei es durch Vorfaelle im direkten Umfeld. sei es durch die Einschraenkung der Wohlfuehlzone auf oeffentlichen Plaetzne im Nahverkehr oder an den schulen der Kinder.

Das bedeutete in den letzten Jahren fuer viele von uns, sich einsetzen zu wollen. Das bedeutete ehrliche Leute fanden etwas, dass sie wert zu verteidigen fanden. Aber das bedeutete auch, dass diese Menschen sich offiziell den Regeln dieser Welt stellen mussten und das veraenderte die Froten.

Es war leicht die Satire im Fernsehen zu akzeptieren, egal wie hetzerisch, stellte sie sich einem Kampf, der einen nicht betraf. Es war leicht rechts als Feindbild zu sehen, als rechts noch besoffene Doerfler und Skinheads bedeutete. Wenn immer mehr Gleichgesinnte beitreten, wird die Hetze schlechter.

Und zwischen all diesem wahnsinn hat es mich wieder zur Politik getrieben. In diesem Aufsatz moechte ich zusammenfassen, wer ich bin und wie es fuer mich dazu kam, mich mit den Identitaere zu identifizieren. Ich beginne also mit einer kurzen Einleitung zu meiner Person. 
es folgt eine detailierte Eroerterung der Dinge, die mciha ufhorchen liessen.
Daraufhin finden sie eine Kerngedanken meiner politischen ansicht zusammengefasst.
Falls Sie das interessiert hat einige Ausblicke und Anstoesse.
Fazit


\section{Meine Geschichte}

Relativ frueh entdeckte ich mein Interesse an politischen Ablaeufen. Dabei war ich zweierlei. Zum einen wie mancher, der frueh zur Politik findet recht radikal und gerechtigkeitsliebend und zum anderen stets an groeseren Ablaeufen, moralischen Pflichten, ethnischen entscheidungen und dem grossen Ganzen interessiert.


\section{Was mich aufmerken lies}

Hielt man mich an unser System und unsere Demokratie zu verteidigen, so waren mir immer zwei Punkte wichtig. Die Freiheit der Presse und die Idee der wehrhaften demokratie und damit ein vertretbarer Pluralismus.
Nachdem ich der Politik und den oeffentlich Medien wieder mehr Aufmerksamkeit schenkte, musste ich beide Ideale mit Fuessen getreten sehen.
jeder Bericht ueber alternative neoliberale Parteien ist hetzerisch und absolut unprofessionell. Es zieht sich von Nachrichten ueber Dokus und Satire bis hin zu einer Folge des Tatorts in der ohne jede rationale Begruendung an einem veralteten Bild der rechten Buerger festgahalten wird.
Das hat mich tief getroffen. denn folgt man ungeschnittenen Interviews dann muss ich damit Menschen diffamiert sehen die ich nicht nur als Mitbuerger sondern als Gleichgesinnte sehe. das ist Volksverrat und aeussert bedenklich.
Den Pluralismus in usnerem system usste ich bedroht sehen als die AfD im Bundestag aus jeglicher Kaolitionsverhandlung kategorisch ausgeschlossen wurde,
15 Prozent im Bundestag zu ignoieren, bedeutet schlichtweg, dass wir 15 Prozent der deutschen das Recht auf Mitbestimmung geben und den rechten mitteilen, dass ihr einziger weg etwas zu aendern bedeutet, dass sie die totale Mehrheit brauchen. Nicht um Uebermass demokratisch. Ausserdem ignorieren wir hier vermutlich die deutsche Mittelschicht.

Lange hatte ich mich aus der Politik herausgehalten, wiel ich keine Bewegung sah, die mich ausreichend vertrat und ferner auch, weil die Probleme die wir hatten von nebensaechlicher Natur waren.

Die meisten Menschen werden nicht von Terroristen getoetet sondern von ihren Bekannten. das ist falsch argumentiert. Wenn jemand sich gegen den Terrorismus stellt, dann ist es kein Argument zu behaupten, dass es ja ein anderes groesseres Problem gibt. Ich denke Stretigkeiten, die dann zu Gewalt und Mord im Bekanntenkreis fuehren, sind ein Problem, das tief in der menshclichen Psyche verankert ist und vielleicht auch im Kern unseres wertesystems. Das entwertet aber in keiner weise die Bedrohung, die durch Terrorismus, islamistische Gewalt und das Aufeinanderprallen von wertesystemen entsteht. Hier wird eine Statistik falsch genutzt.

Zitat Niko Semsrott: der Aufklaerer sagt: Habt keine Ansgt vor Terroristen sondern vor der eigenen Familie. 
Das ist so nicht korrekt. Der Aufklaerer sollte aufrufen beide Probleme als real zu betrachten. Der noch kluegerer Aufklaerer wuerde erkennen, dass die Angst vor dem Terrorismus kein rationales Problem ist. Also sollte er sich einsetzen um die Menschen zu beruhigen und so das Problem zu loesen. Menschen in Angst sind nun mal nciht rational
Ausserdem ist das Problem von Gewalt im Bekanntenkreis ein problem, dem man sich selber stellen kann und muss, indem man vernuenftig miteinander umzugehen versucht. Der Terrorismus und die Gewalt die Fluechtlinge in dieses Land bringen ist ein abstraktes Problem, gegen das der einzelen machtlos ist.

Vergleich Fluechtlinge mit Polen die den Spargel stechen
Hauptgegenargument: Der Pole kommt hier her um zu arbeiten
Nebenargumente: Der Pole teilt einen grossteil unserer werte, der pole liefert keinen anlass ihn zu fuerchten

Wir sprechen immer wiede von der gefaehlrichen Ideologie der AfD. Ich verstehe nicht ausschliesslich warum diese Ideologie gefaehrlich ist.

Die AfD sind schlechte Menschen, weil sie Leute an den Grenzen erschiessen sollen. Gegen das Argument ist so nichts zu sagen. Das Moralsystem, das wir aufgebaut haben, sieht das Toeten als schlecht an aber unser Moralsystem hat auch den staat erschaffen, in den nun alle fliehen wollen und der Gegenpunkt zur Moral ist immer der Utilitarismus. das ist weshalb wir ethnik kommisionen haben und keinen schreienenden Peobel. Wenn wir weiter Menschen helfen wollen, dann muessen wir den Kern des systems schuetezen

Sarah Lesch distanziert sich von rechts. Warum haben wir keine Bewegung, die die Frauen moegen kann.

\section{Mein Loesungsvorschlag}

At this point, you've changed everything (including your marks!). Time to wrap up!

\section{Was Sie vielleicht interessieren koennte}


\section{Fazit}


\clearpage
\printbibliography

\end{document}